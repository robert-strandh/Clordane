\chapter{Backtrace}
\label{chap-backtrace}

When an unhandled error is signaled in the application thread, the
execution of that thread is stopped, and \sysname{} displays a
\emph{backtrace}.  By default, the backtrace shows only function calls
that are part of the application itself, and not part of the internal
workings of the \commonlisp{} system.  A button can be clicked on to
toggle this setting so that all stack frames are shown.

Clicking on one of the stack frames makes this stack frame the
\emph{current} one.  A different stack frame can also be selected by
navigating with the keystrokes \texttt{p} (for \emph{previous}) and
\texttt{n} (for \emph{next}).  The \sysname{} REPL can then be used to
evaluate forms in the lexical environment of that stack frame, and the
\texttt{*package*} special variable is set to the package in which the
code of the current stack frame has been defined.

Initially, the \emph{stoppingpoint{} window} discussed in
\refSec{sec-windows-stopping-point} shows the point in the code where
the error was signaled.  When the user selects some stack frame, the
stoppingpoint{} window shows the code around the point where execution
will resume if control is returned to that stack frame.  As before,
the value of live variables can be obtained by hovering the pointer
over a variable.
