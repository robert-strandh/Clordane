\chapter{Implementing poll points}
\label{app-implementing-poll-points}

There are many ways of implementing poll points.  However, we consider
that a reasonable implementation must have the following
characteristics:

\begin{itemize}
\item There should be little or no performance penalty for code that
  is running in some thread other than a thread being debugged.
\item There should be little performance penalty for poll points that
  are not currently breakpoints.
\end{itemize}

We suggest the following technique which will work for any processor
architecture:

\begin{itemize}
\item The thread instance contains three pieces of information
  associated with poll points:
  \begin{enumerate}
  \item A \emph{debug flag} which is a single bit indicating whether
    the thread is running under the control of the debugger.  At the
    beginning of each top-level function compiled with a high value of
    the \texttt{debug} quality, this flag is loaded into a lexical
    variable, subject to register allocation as usual.
  \item A \emph{breakpoint table} (perhaps a hash table) indexed by values of
    the program counter, and containing information about the nature
    of a breakpoint at this program point.
  \item A small \emph{bit table}, say 1024 bits or so.  The index into
    this table is the value of the program counter modulo its size.
    It contains summary information from the previous table, i.e.,
    whenever some value of the program counter is present in the
    breakpoint table, then the corresponding bit is set in this bit
    table.
  \end{enumerate}
\item When code is compiled with a high value of the \texttt{debug}
  quality, a small, local routine is added to each top-level
  function.  This routine is called using an inexpensive call
  protocol, typically just a \texttt{jsr} instruction.  The routine
  performs the following actions:
  \begin{itemize}
  \item It starts by checking the debug flag that was loaded into a
    lexical variable from the thread instance of the running thread.
    If that flag is cleared, then the routine returns without any
    action.
  \item If the flag is set, then it uses the value of the program
    counter stored on the stack as part of the call protocol.  It
    takes that value modulo the size of the bit table and checks
    whether the entry in the bit table is set.  If it is cleared,
    again, the routine returns without any action.
  \item If the bit-table entry is set, then it consults the breakpoint
    table to see whether the actual value of the program counter is
    present there.  If not, the routine again returns normally.  If it
    is present, then it suspends the execution of the thread and hands
    over control to the debugger thread.
  \end{itemize}
\item Also when code is compiled with a high value of the
  \texttt{debug} quality, a call to the small routine is inserted
  before the beginning, and after the end of the execution of every
  source-level form present in the code.
\end{itemize}

The debugger maintains the breakpoint table and the bit table as
follows:

\begin{itemize}
\item When a breakpoint is set, it is added to the breakpoint table,
  and bit in the bit table corresponding to the value of the program
  counter modulo the size of the bit table is set.
\item When a breakpoint is removed, the entire bit table is first
  cleared.  Then the breakpoint table is traversed and, for each
  breakpoint, the corresponding bit in the bit table is set as
  before.
\end{itemize}
