\chapter{Introduction}
\pagenumbering{arabic}%

\section{Purpose}

\sysname{} is a debugger for \commonlisp{}.  The \commonlisp{}
standard has an entry for the word ``debugger'' in the glossary, but
there are no requirements on the capabilities of a debugger there,
other than that it should allow the user to handle conditions
interactively.  Most existing implementations provide more
functionality, such as the ability to examine active stack frames
including the values of live variables and the source location of the
return address of a particular stack frame.

In this document, we define a debugger with many more features than
what is typically provided.  In particular, we take advantage of the
existence of \emph{threads} in most modern \commonlisp{}
implementations to define improvements to the operations provided by
the debugger.

The downside of these additional features is that \sysname{} will
require much more assistance from the supported implementations than
required by existing debuggers.

\section{Terminology}

%
\def\Applicationthread{Application thread}%
\def\applicationthread{application thread}%
\def\applicationthreads{application threads}%
\subsection{\Applicationthread{}}

From the point of view of \sysname{}, the \applicationthread{} is the
thread that the programmer wants to debug.  Other threads running the
same code as the \applicationthread{} are not affected by the
debugging actions taken by the programmer.

The \applicationthread{} can run any code, including that of
\sysname{}.

%
\def\Debuggerthread{Debugger thread}%
\def\debuggerthread{debugger thread}%
\def\debuggerthreads{debugger threads}%
\subsection{\Debuggerthread{}}

The \debuggerthread{} is the thread running \sysname{}.  The
\debuggerthread{} can not debug itself, but it can debug the same code
running in a different thread.

%
\def\Pollpoint{Poll point}%
\def\pollpoint{poll point}%
\def\pollpoints{poll points}%
\subsection{\Pollpoint{}}

A \emph{\pollpoint{}} is a place in the program where a running thread
can be \emph{stopped}.

From the point of view of the compiler, a \pollpoint{} is a place
where code is inserted so that the program interrogates its
\emph{thread} to see whether it should take some action, such as
generating trace output, stopping the execution, or some other action.
For a possible implementation of poll points, see
\refApp{app-implementing-poll-points}.

The compiler generates \pollpoints{} only where it is \emph{safe} to
stop the program.

Compiling code with a higher \texttt{debug} value gives executable
code with more \pollpoints{}.

%
\def\Breakpoint{Break point}%
\def\breakpoint{break point}%
\def\breakpoints{break points}%
\subsection{\Breakpoint{}}

A \emph{\breakpoint{}} is a \pollpoint{} that has been marked to indicate
that the running program should stop its execution when it is
reached.

Such \breakpoints{} can be \emph{created} or \emph{destroyed} for a
thread by \sysname{} independently of whether the thread is currently
executing or currently stopped.

A \breakpoint{} can be \emph{steady} or \emph{volatile}.  A
\emph{steady} \breakpoint{} is destroyed only as a result of an action
on the part of the programmer.  A \emph{volatile} \breakpoint{} is
also destroyed when the program control reaches it.

A \breakpoint{} can be associated with a \commonlisp{} \emph{form} to
be evaluated when the \breakpoint{} is reached.  The form is evaluated
by \sysname{}.  Tracing can be accomplished by associating with the
\breakpoint{} a form that will print some information and then
\emph{continue} the execution of the thread.

%
\def\Stoppingpoint{Stopping point}%
\def\stoppingpoint{stopping point}%
\def\stoppingpoints{stopping points}%
\subsection{\Stoppingpoint{}}

A \emph{\stoppingpoint{}} is a place in the program where the thread
of execution is currently stopped.  A \stoppingpoint{} can be any
program counter value, but a value of the program counter other than
a \pollpoint{} can not be reached by creating a \breakpoint{}.  It
can, however, be reached by \emph{stepping by instruction} after the
program has stopped at a \pollpoint{}.

%
\def\Continue{Continue}%
\def\continue{continue}%
\def\continues{continues}%
\subsection{\Continue{}}

One action the programmer can take when the \applicationthread{} is
stopped is to instruct \sysname{} to \emph{\continue{}} the execution
of the \applicationthread{}.  The execution of the
\applicationthread{} will then resume until it either terminates or
reaches a \breakpoint{}.

%
\def\Dvance{Advance}%
\def\dvance{advance}%
\def\dvances{advances}%
\subsection{\Dvance{}}

When the \applicationthread{} is stopped, the application programmer
can instruct \sysname{} to \emph{\dvance{}} to a particular
\pollpoint{}.  \sysname{} will then insert a \emph{volatile
  \breakpoint{}} at that point and then \continue{} the execution of
the \applicationthread{}.

%
\def\Step{Step}%
\def\step{step}%
\def\steps{steps}%
\subsection{\Step{}}

The application programmer can instruct \sysname{} to \emph{\step{}}
the execution of the \applicationthread{}, either by \emph{\pollpoint}
or by \emph{instruction}.  

When \sysname{} is instructed to \step{} by \pollpoint{}, it will
continue the execution of the application thread until the next
\pollpoint{} is reached, at which point \sysname{} will \emph{stop}
the \applicationthread{}, without setting any \breakpoint{}. 

When \sysname{} is instructed to \step{} by instruction, it will
execute the next machine instruction and then stop the execution of
the \applicationthread{}.
