\chapter{Stepping}

Stepping refers to the action of making the execution of the debugged
program advance by a specific unit of code.  There are several ways
for the programmer to determine the unit of code to step by.

\section{Step over}

The action called \emph{step over} works as follows:

\begin{itemize}
\item If the current stopping point is at the end of the last sub-form
  of its immediately enclosing form, then the \emph{step over} command
  has no effect, so the stopping point remains the same.
\item If the current stopping point is at the beginning of a form $F$
  to be evaluated, and $F$ is the last sub-form of its immediately
  enclosing form, then the next stopping point is at the end of $F$.
\item If the current stopping point is at the beginning of a form $F$
  to be evaluated, and there is only one possible next form $G$ to
  evaluate, then $F$ is evaluated and the next stopping point is at
  the beginning of $G$.
\item If the current stopping point is at the beginning of a form $F$
  to be evaluated, and there are two possible next forms $G1$ and $G2$
  depending on the value of $F$, then $F$ is evaluated and the next
  stopping point is ether at the beginning of $G1$ or at the beginning
  of $G2$, according to the value of $F$.
\end{itemize}

When a form $F$ is evaluated as a result of the \emph{step over} command,
then if $F$ executes a non-local transfer of control such as a
\texttt{return-from}, a \texttt{go}, or a \texttt{throw}, then
execution may or may not stop at the next stopping point.  Execution
will stop only of some stopping point is encountered.

\section{Step in}

The action called \emph{step in} works as follows:

\begin{itemize}
\item If the current stopping point is at the end of the last sub-form
  $F$ of its immediately enclosing form $G$, and $G$ is \emph{not} a
  function call, then the \emph{step in} command has no effect, so the
  stopping point remains the same.
\item If the current stopping point is at the end of the last sub-form
  $F$ of its immediately enclosing form $G$, and $G$ \emph{is} a
  function call, then the next stopping point is at the beginning of
  the first form to be evaluated in the function being called.
\item If the current stopping point is at the beginning of a form $F$
  to be evaluated, and $F$ has sub-forms that need evaluation, then
  the next stopping point is at the beginning of the first sub-form
  $G$ of $F$ that needs evaluation.
\item If the current stopping point is at the beginning of a form $F$
  to be evaluated, and $F$ has no sub-forms that need evaluation, then
  the next stopping point is at the end of the last sub-form of $F$.
\end{itemize}

\section{Step out}

The action called \emph{step out} works as follows:

\begin{itemize}
\item If the current stopping point is at the beginning or at the end
  of a form $F$ whose immediately enclosing form is a function
  definition, then the next stopping point is the same as if a
  \emph{step over} command had been executed at the beginning of the
  corresponding function-call form in the caller.
\item If the current stopping point is at the beginning or at the end
  of a form $F$ whose immediately enclosing form is $G$ that is
  \emph{not} a function definition, then the next stopping point is
  the same as if a \emph{step over} command had been executed at the
  beginning of $G$.
\end{itemize}

