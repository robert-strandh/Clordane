\chapter{Stepping}

Stepping refers to the action of making the execution of the debugged
program advance by a specific unit of code.  There are several ways
for the programmer to determine the unit of code to step by.

\section{Step over}

The action called \emph{step over} works as follows:

\begin{itemize}
\item If the current stopping point is at the end of the last sub-form
  of its immediately enclosing form, then the \emph{step over} command
  has no effect, so the stopping point remains the same.
\item If the current stopping point is at the beginning of a form $F$
  to be evaluated, and $F$ is the last sub-form of its immediately
  enclosing form, then the next stopping point is at the end of $F$.
\item If the current stopping point is at the beginning of a form $F$
  to be evaluated, and there is only one possible next form $G$ to
  evaluate, then $F$ is evaluated and the next stopping point is at
  the beginning of $G$.
\item If the current stopping point is at the beginning of a form $F$
  to be evaluated, and there are two possible next forms $G1$ and $G2$
  depending on the value of $F$, then $F$ is evaluated and the next
  stopping point is ether at the beginning of $G1$ or at the beginning
  of $G2$, according to the value of $F$.
\end{itemize}

\section{Step in}

The action called \emph{step in} works as follows:

\begin{itemize}
\item If the current stopping point is at the beginning of a form $F$
  to be evaluated, and $F$ has sub-forms that need evaluation, then
  the next stopping point is at the beginning of the first sub-form
  $G$ of $F$ that needs evaluation.
\item If the current stopping point is at the beginning of a form $F$
  to be evaluated, and $F$ has no sub-forms that need evaluation, then
  the next stopping point is at the end of $F$.
\item If the current stopping point is at the end of a form $F$ to be
  evaluated, then this action works the same way as \emph{step over}.
\end{itemize}

\section{Step out}

The action called \emph{step out} works as follows:
[to be filled in]

