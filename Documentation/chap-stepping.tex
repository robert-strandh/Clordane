\chapter{Stepping}

Stepping refers to the action of making the execution of the debugged
program advance by a specific unit of code.  There are several ways
for the programmer to determine the unit of code to step by:

\begin{itemize}
\item If the current stopping point is located immediately
  \emph{before} some expression, then the step may execute that
  expression and then stop when that expression is completely
  evaluated.  This action is called \emph{step over}.
\item If the current stopping point is located immediately
  \emph{before} some expression, then the step may execute that
  expression and then stop immediately before the next expression to
  be evaluated.  This action is called \emph{step next}
\item If the current stopping point is located immediately
  \emph{after} some expression, then the step may result in no
  execution, and then stop immediately before the next expression to
  be evaluated.  This action is called \emph{step null}
\item If the current stopping point is located immediately
  \emph{after} some expression, then the step may execute the next
  expression to be evaluated and then stop immediately after that
  expression.  This action is called \emph{step next} and is
  distinguished from the previous action by the initial position of
  the stopping point.  If the stopping point is both immediately after
  some expression and immediately before the following expression,
  then this action is equivalent to \emph{step over}.
\item If the current stopping point is located immediately
  \emph{before} some expression and that expression is a function
  call, then the step may be to enter the called function and stop
  before the first expression of that function is evaluated.  This
  action is called \emph{step in}
\item The remaining expressions of the currently executing function
  may be evaluated and execution stopped immediately after that
  function returns to its caller.  This action is called \emph{step
    finish}.
\item The step could be a single machine instruction in which case the
  next instruction is executed and then the execution is stopped.
  This action is called \emph{step instruction}.
\end{itemize}
