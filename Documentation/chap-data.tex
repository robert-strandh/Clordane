\chapter{Examining and modifying data}

When the application is stopped at some stopping point, \sysname{}
provides a REPL for the user.  This REPL is not the ordinary REPL of
the \commonlisp{} implementation, in that it operates in the lexical
environment of the stopping point.  Also, the available functionality
of the REPL may be restricted so as to preserve the integrity of the
system.

In particular, a references to a lexical variable is resolved to the
place where the compiler allocated space for it, which might be in a
register or on the stack.  The set of available such references may
depend on the \texttt{debug} level when the code was compiled.  The
application programmer may use \texttt{setq} to alter the value of
lexical variables.  The value to be assigned is then restricted to the
\emph{type} that the compiler inferred for this variable so that the
code behaves in a consistent way after an assignment.  Implementations
may of course choose to disable type inference so that any value is
allowed, or they may disallow references to lexical variables for
which they have no \sysname{} assignment support.

In addition to providing a REPL, \sysname{} automatically displays all
available lexical variables at the stopping point, allowing the
application programmer to see which variables can be referred to.
Each variable is shown with its name, its value (which may be clicked
on for inspection), and its type.
