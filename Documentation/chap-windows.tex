\chapter{Different types of windows}

\section{Source window}
\label{sec-windows-source}

A source window is a window that displays source code together with
\pollpoints{}.  A \pollpoint{} which is not a \breakpoint{} is shown
as a cursor with blue color.  A \pollpoint{} which is also a
\breakpoint{} is shown as a cursor with red color.

\section{\Stoppingpoint{} window}
\label{sec-windows-stopping-point}

A \stoppingpoint{} window is a window that is popped up whenever 
the \applicationthread{} is stopped.

The \stoppingpoint{} window shows the position of the \stoppingpoint{}
and variables that are \emph{live} at the \stoppingpoint{}.  The value
of a live variable can be obtained by hovering the pointer over the
variable.

In the \stoppingpoint{} window is also shown \pollpoints{} in the
source code to which the user can ask the program to \emph{\dvance{}}.
Clicking with the left mouse button on such a point will set a
temporary \stoppingpoint{} at that position and then the program will
continue from the \stoppingpoint{}.  The temporary \stoppingpoint{} is
removed as soon as it is reached.

In this window, there are button that the user can click on.  One such
button is marked \emph{Finish} and results in the program continuing
from the \stoppingpoint{} to immediately after the next \emph{return}.

Another button is marked \emph{Enter}.  This button is clickable only
if the \stoppingpoint{} immediately precedes a function call or a tail
call.  Clicking this button sets a temporary \stoppingpoint{} at the
start of the function about to be called and then the program
execution continues from the \stoppingpoint{}.

\section{Stack backtrace window}

Whenever the program is stopped, the \emph{stack backtrace window} is
updated.  See \refChap{chap-backtrace} for more information on the
backtrace facility.
